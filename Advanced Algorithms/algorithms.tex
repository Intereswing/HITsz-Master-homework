\documentclass{article}
\usepackage{amsmath}
\usepackage[ruled]{algorithm}
\usepackage{algorithmic}

\title{Advanced Algorithms Homework}
\author{Intereswing}

\begin{document}
\maketitle

\section{Lecture 8: Augmenting Data Structures}
\paragraph{1}

Insertion into a red-black tree consists of two phases. The first phase inserts the new node, and the second phase maintaining the red-black properties, which may perform rotations. 

To maintain the rank of subtree in the first phase, simply increment $x.rank$ for each node $x$, whose left child is also on the simple path traversed from the root down towards the leaves.

In the second phase, the only structural changes to the underlying red-black tree are caused by rotations. Moreover, only one node have its $rank$ attributes invalidated in a rotation. Referring to the code for $\textsc{Left-Rotate}(T,x)$ on page 336 in \textit{Introduction to Algorithms}, add the following line:

\addtocounter{algorithm}{1}
\begin{algorithm}
\begin{algorithmic}[1]
	\STATE \(y.rank = x.rank + y.rank\)
\end{algorithmic}
\end{algorithm}






















\end{document}